% This is "sig-alternate.tex" V1.9 April 2009
% This file should be compiled with V2.4 of "sig-alternate.cls" April 2009
%
% This example file demonstrates the use of the 'sig-alternate.cls'
% V2.4 LaTeX2e document class file. It is for those submitting
% articles to ACM Conference Proceedings WHO DO NOT WISH TO
% STRICTLY ADHERE TO THE SIGS (PUBS-BOARD-ENDORSED) STYLE.
% The 'sig-alternate.cls' file will produce a similar-looking,
% albeit, 'tighter' paper resulting in, invariably, fewer pages.
%
% ----------------------------------------------------------------------------------------------------------------
% This .tex file (and associated .cls V2.4) produces:
%       1) The Permission Statement
%       2) The Conference (location) Info information
%       3) The Copyright Line with ACM data
%       4) NO page numbers
%
% as against the acm_proc_article-sp.cls file which
% DOES NOT produce 1) thru' 3) above.
%
% Using 'sig-alternate.cls' you have control, however, from within
% the source .tex file, over both the CopyrightYear
% (defaulted to 200X) and the ACM Copyright Data
% (defaulted to X-XXXXX-XX-X/XX/XX).
% e.g.
% \CopyrightYear{2007} will cause 2007 to appear in the copyright line.
% \crdata{0-12345-67-8/90/12} will cause 0-12345-67-8/90/12 to appear in the copyright line.
%
% ---------------------------------------------------------------------------------------------------------------
% This .tex source is an example which *does* use
% the .bib file (from which the .bbl file % is produced).
% REMEMBER HOWEVER: After having produced the .bbl file,
% and prior to final submission, you *NEED* to 'insert'
% your .bbl file into your source .tex file so as to provide
% ONE 'self-contained' source file.
%
% ================= IF YOU HAVE QUESTIONS =======================
% Questions regarding the SIGS styles, SIGS policies and
% procedures, Conferences etc. should be sent to
% Adrienne Griscti (griscti@acm.org)
%
% Technical questions _only_ to
% Gerald Murray (murray@hq.acm.org)
% ===============================================================
%
% For tracking purposes - this is V1.9 - April 2009

\documentclass{sig-alternate}

\begin{document}
%
% --- Author Metadata here ---
\conferenceinfo{OzCHI}{'15, December 07-10, 2015, Melbourne, Australia}
%\CopyrightYear{2015} % Allows default copyright year (20XX) to be over-ridden - IF NEED BE.
%\crdata{0-12345-67-8/90/01}  % Allows default copyright data (0-89791-88-6/97/05) to be over-ridden - IF NEED BE.
% --- End of Author Metadata ---

\title{Comparing usability of concurrency models}

\numberofauthors{1}
\author{
\alignauthor
Ben Swift, Peter Davis, Henry Gardner\\
\vskip .3em
\affaddr{Research School of Computer Science}\\
\affaddr{Australian National University}\\
\affaddr{Canberra, Australia}\\
\vskip .3em
\email{\{ben.swift,peter.davis,henry.gardner\}@anu.edu.au}
}

\maketitle

\begin{abstract}
Abstract goes here.
\end{abstract}

\section{Introduction to Concurrency}
Concurrency refers to systems where there is more than one process existing at a time, whose component processes interact with each other by communication \cite{tpc} . It does not say anything about the processes executing at the same time, and concurrent systems on single processor machines are an example of this.

Asynchronous definition

Parallel definition

Distributed definition

\subsection{Concurrency Problems}
Deadlock

Livelock

Thrashing

\subsection{Concurrency layers}
Introduce 'axes', with which concurrency models will be evaluated.

Similar to the OSI model of splitting the network stack by each layer, each layer relies on the previous layer in order to function properly and provide services to the next layer up.

\begin{table}[]
\centering
\caption{My caption}
\label{my-label}
\begin{tabular}{llll}
 & Layer & Description & Example \\ \hline
1 & Physical & Layer responsible for items such as: location, power, cooling, physical connections between machines & Physical computing hardware, network connnections \\
2 & Communication & Layer responsible for allowing any processes to communicate in a common fashion & MPI, ZeroMQ, Pointers \\
3 & Concurrency & Layer responsible for organising concurrent execution of code. & BSP, Actor, CSP, Threads \\
4 & Algorithmic Skeleton/Topology & Layer responsible for organising and directing the interconnections between concurrent processes. This may correspond to the physical layout of the machines. & Farm, pipe, map, divide and conquer\\
5 & Solution Algorithm & Layer responsible for providing a solution to the problem being tasked. & FFT, MapReduce, simulations
\end{tabular}
\end{table}

\section{Usability in a Programming Context}
Need some programming language usability metrics which are shown to improve software quality.

\section{Concurrency Models}
\subsection{Threads}


\subsection{BSP}
Original paper advocated distributed memory using hashing
Modern implementations such as Pregel and Apache Giraph use a messaging system.
Supersteps within which it is assumed that no synchronisation is required
End of superstep a global synchronisation step

\subsection{Actor}
Independent workers
Messaging system, mailboxes

\subsection{CSP}
Go channels and clojures core.async

\subsection{}

\section{Comparing Models}

\section{Conclusion}



% \begin{figure}
% \centering
% \epsfig{file=fly.eps, height=1in, width=1in}
% \caption{A sample black and white graphic (.eps format)
% that has been resized with the \texttt{epsfig} command.}
% \end{figure}

\bibliographystyle{abbrv}
\bibliography{distributed-computing-ozchi-2015}
\end{document}

%%% Local Variables:
%%% mode: latex
%%% TeX-master: t
%%% End:
